\documentclass[aspectratio=43]{beamer}
\usetheme{Boadilla}
%\usepackage{slashbox}
%\usepackage{hanging}
%\usepackage{algorithm,algorithmic}
\usepackage{amsmath}
\usepackage{tikz}
\usepackage{fancybox}
\usepackage{mathrsfs}
\usepackage{makecell}
\usepackage{listings}
\usepackage{url}
%\usepackage[rightcaption]{sidecap}
%\usepackage{enumitem}
% set first line indentation
\parindent=20pt
\setbeamersize{text margin left=20pt,text margin right=20pt}

\logo {\includegraphics[height=0.8cm]{./hitlogo.eps}}
% items enclosed in square brackets are optional; explanation below
\title[Salt]{Salt detection with deep learning}
%\subtitle[subs]{Parallelization with domain decomposition}
\author[W. Wang]{Wenlong Wang and Fangshu Yang}
\titlegraphic{\includegraphics[height=0.8cm]{./hitlogo.eps}}

\institute[HIT]{Harbin Institute of Technology}
\date[Nov 2017]{Nov, 2017}

\begin{document}

% remove figure caption
\setbeamertemplate{caption}{\insertcaption}

%\frame{
%\begin{figure}
%\rule{5cm}{5cm}
%\caption{Test}
%\end{figure}
%}


%--- the titlepage frame -------------------------%
\begin{frame}[plain]
  \titlepage
\end{frame}

\begin{frame}
\frametitle{Outline}
\tableofcontents
%\tableofcontents[part=1,pausesections]
\end{frame}
%-----------------------------------------------------------
\section{Motivations}
\begin{frame}{Motivations}
\begin{itemize}
\item{Salts are ideal for seal, thus are closely related with reservoirs.}
\item{Traditional migration and inversion don't distinguish salt reflections and other reflections.}
\item{Salt detection requires interpretations with heavy human efforts.}
\item{We want to use deep learning to eliminate human efforts and automatically detect salt from seismic traces}
\end{itemize}
\end{frame}
%-----------------------------------------------------------
\begin{frame}{Previous works}
\begin{itemize}
\item{Jing et al., 2007, propose to compute amplitude discontinuity attributes using edge-detection techniques.}
\item{Bertholot et al., 2013 compute texture attributes.}
\item{Halpert and Clapp, 2008, calculate seismic reflection dip attributes.}
\item{Amin and Deriche, 2015, suggest using multiple attributes combined for highlighting salt boundaries.} 
\end{itemize}
\end{frame}
%-----------------------------------------------------------
\begin{frame}{The aim}
\begin{itemize}
\item{Input: seismic trace(s) (data domain)}
\item{Output: salt segmentation (space domain)}
\end{itemize}
\noindent{The major difficulty is the transformation from data domain (x-t) to space domain (x-z).}
\end{frame}
%-----------------------------------------------------------
\section{The network}
\begin{frame}
\frametitle{Outline}
\tableofcontents[currentsection]
\end{frame}
\begin{frame}{The network}
    \includegraphics[width=\textwidth]{./sketch.eps}
\end{frame}
%-----------------------------------------------------------
\begin{frame}{Network architecture}
\begin{itemize}
\item{Training size: 1000}
\item{Testing size: 20}
\item{Num of Epochs: 200}
\item{Batch size: 10}
\item{Num of classes: 2}
\item{Model generation: arbitrary increasing velocity profile (2500-3500 m/s) with a convex-shaped salt (4500 m/s)}
\end{itemize}
\end{frame}
%-----------------------------------------------------------
\begin{frame}{Training Samples}
    \includegraphics[width=\textwidth]{./model_sample.eps}
\end{frame}
%-----------------------------------------------------------
\begin{frame}{Learning curve}
    \includegraphics[width=\textwidth]{./cost_function.eps}
\end{frame}
%-----------------------------------------------------------
\begin{frame}{Test results (1 shot as input)}
    \includegraphics[width=\textwidth]{./test_sample1.eps}
\end{frame}
%-----------------------------------------------------------
\begin{frame}{Test results (10 shots as input)}
    \includegraphics[width=\textwidth]{./test_sample.eps}
\end{frame}
%-----------------------------------------------------------
\begin{frame}{1 shot vs. 10 shots training}
\begin{table}[]
\centering
\label{my-label}
\begin{tabular}{|l|l|l|}
\hline
         & IOU    & Time (s/epoch) \\ \hline
1 shot   & 90.1\% & 18.70          \\ \hline
10 shots & 94.7\% & 55.82          \\ \hline
\end{tabular}
\end{table}
\end{frame}
%-----------------------------------------------------------
\section{Deep learning vs. FWI}
\begin{frame}
\frametitle{Outline}
\tableofcontents[currentsection]
\end{frame}

\begin{frame}{Net architecture}
\begin{itemize}
\item{Training size: 1000}
\item{Testing size: 20}
\item{Num of Epochs: 200}
\item{Batch size: 10}
\item{Num of classes: 2}
\item{Channel size: 32}
\item{Model generation: arbitrary convex-shaped anormaly (2500 m/s) with homogeneous background velocity 2300 m/s}
\end{itemize}
\end{frame}
%-----------------------------------------------------------
\begin{frame}{}
 \centering
 \includegraphics[width=1.0\textwidth]{/data/fcn_seis/fcn/labelfigs/figure1.eps}
\end{frame}
%-----------------------------------------------------------
\begin{frame}{}
 \centering
 \includegraphics[width=1.0\textwidth]{/data/fcn_seis/fcn/labelfigs/figure2.eps}
\end{frame}
%-----------------------------------------------------------
\begin{frame}{}
 \centering
 \includegraphics[width=1.0\textwidth]{/data/fcn_seis/fcn/labelfigs/figure3.eps}
\end{frame}
%-----------------------------------------------------------
\begin{frame}{}
 \centering
 \includegraphics[width=1.0\textwidth]{/data/fcn_seis/fcn/labelfigs/figure4.eps}
\end{frame}
%-----------------------------------------------------------
\begin{frame}{}
 \centering
 \includegraphics[width=1.0\textwidth]{/data/fcn_seis/fcn/labelfigs/figure5.eps}
\end{frame}
%-----------------------------------------------------------
\begin{frame}{}
 \centering
 \includegraphics[width=1.0\textwidth]{/data/fcn_seis/fcn/labelfigs/figure51.eps}
\end{frame}
%-----------------------------------------------------------
\begin{frame}{DL vs. FWI}
\begin{table}[]
\centering
\label{my-label}
\begin{tabular}{|l|l|l|}
\hline
                     & Training & Processing \\ \hline
FWI (MPI 30 cores)   & N/A     & 30 mins          \\ \hline
DL (GPU)             & 60 mins & $<$ 1 s         \\ \hline
\end{tabular}
\end{table}
\end{frame}
%-----------------------------------------------------------
\section{Deep learning for FWI}
\begin{frame}
\frametitle{Outline}
\tableofcontents[currentsection]
\end{frame}
\begin{frame}
How about regression?
\end{frame}
%-----------------------------------------------------------
\begin{frame}{}
 \centering
 \includegraphics[width=1.0\textwidth]{/data/fcn_seis/fcn/figure1.eps}
\end{frame}
%-----------------------------------------------------------
\begin{frame}{}
 \centering
 \includegraphics[width=1.0\textwidth]{/data/fcn_seis/fcn/figure2.eps}
\end{frame}
%-----------------------------------------------------------
\begin{frame}{}
 \centering
 \includegraphics[width=1.0\textwidth]{/data/fcn_seis/fcn/figure3.eps}
\end{frame}
%-----------------------------------------------------------
\begin{frame}{}
 \centering
 \includegraphics[width=1.0\textwidth]{/data/fcn_seis/fcn/figure4.eps}
\end{frame}
%-----------------------------------------------------------
\begin{frame}{}
 \centering
 \includegraphics[width=1.0\textwidth]{/data/fcn_seis/fcn/figure5.eps}
\end{frame}
%-----------------------------------------------------------
\section{Discussions}
\begin{frame}{Discussions}
\begin{itemize}
\item{The deep network (FCN) seems capable of understanding the transformation from data to space domain.}
\item{Once the model is trained, the salt detection process is completely data-driven, no human efforts are involved.}
\item{Real data application is still challenging with limited training samples.}
\item{Another limitation is that the network also requires fixed geometry for each model.}
\end{itemize}
\end{frame}
%%-----------------------------------------------------------
\begin{frame}
\begin{center}
  \includegraphics[width=0.6\textwidth]{./machine_learning.eps}
\end{center}
\end{frame}

\end{document}

